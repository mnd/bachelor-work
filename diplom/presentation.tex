% -*- mode: LaTeX; mode: auto-fill; mode: flyspell; coding: utf-8; -*-
\documentclass[ucs]{beamer}
\usepackage[T2A]{fontenc}
\usepackage[english,russian]{babel}
\usepackage[utf8x]{inputenc}
\usepackage{verbatim}
\newenvironment{code}{\small\verbatim}{\normalsize\endverbatim}

\usetheme % [secheader]
    {Madrid}

\title[Динамическая диспетчеризация]{Динамическая диспетчеризация в языке Haskell}
\author{Меринов Николай}
\institute{УрГУ}
\date{2010}
\begin{document}

\begin{frame}
\titlepage
\end{frame}

\begin{frame}[fragile]
  \frametitle{Статические системы типов}
  \framesubtitle{Преимущества}
  \begin{block}{Документация}
    %% Указание типов переменных позволяет судить о том что делает код
\begin{verbatim}
filter :: (a -> Bool) -> [a] -> [a]
\end{verbatim}
  \end{block}

  \begin{block}{Контроль ошибок}
    %% Попытка вызвать функцию \verb=square= от нечисловой переменной
    %% будет обнаружена на этапе компиляции
\begin{verbatim}
square :: Integer -> Integer
func = square "test"
\end{verbatim}
  \end{block}

  \begin{block}{Простота изменения кода}
    %% В случае изменения типа данных Man компилятор укажет все места
    %% которых коснулось изменение
\begin{verbatim}
data Man = P Name Gender Age

read :: String -> Man
process :: Man -> Man
\end{verbatim}
  \end{block}
\end{frame}

\begin{frame}[fragile]
  \frametitle{Статические системы типов}
  \framesubtitle{Недостатки}

  \begin{block}{Усложнение синтаксиса}
    %% Добавление в язык новых конструкций, иногда приходится писать
    %% как в примере.
\begin{verbatim}
(square :: (Double -> Double)) (1 :: Double)
\end{verbatim}
  \end{block}

  \begin{block}{Уменьшение выразительности}
    %% Невозможно выразить такое простое поведение как в классическом
    %% примере на perl
\begin{verbatim}
open(F, "filename") or die("can't open");
\end{verbatim}
  \end{block}

  \begin{block}{Монолитность решений}
    %% При добавлении нового конструктора в data Expr придётся менять
    %% все функции обрабатывающие выражение непосредственно там где
    %% они написаны.
\begin{verbatim}
data Expr = I Int | D Dobule | Sum Expr Expr

inc :: Expr -> Expr
inc (I a) = a + 1
inc (D a) = a + 1.0
inc (S a b) = S (inc a) (inc b)
\end{verbatim}
  \end{block}
\end{frame}

\begin{frame}[fragile]
  \frametitle{Борьба с недостатками}
  \framesubtitle{Генерация кода}

  \begin{block}{Текстовая подстановка}
    %% C-like макросы
\begin{verbatim}
#define INSTANCE_SHOW(tycon, str) \
instance Show tycon where { show _ = str }

INSTANCE_SHOW(Type1, "Type1 object")
\end{verbatim}
  \end{block}

  \begin{block}{Template Haskell}
    %% генерация кода программой на Haskell
\begin{verbatim}
sumN n = do
  sum <- [| sum |]
  names <- mapM newName $ take n $ repeat "var"
  return $ LamE (map VarP names)
    (AppE sum (ListE (map VarE names)))

$(sumN 3) 1 2 3 == 6
\end{verbatim}
  \end{block}
\end{frame}

\begin{frame}[fragile]
  \frametitle{Борьба с недостатками}
  \framesubtitle{Добавление исключений в систему типов}

  \begin{block}{Включение в язык функций соответствующих любому типу}
\begin{verbatim}
if isDataCorrect a
  then parse a
  else error "incorrect input"
\end{verbatim}
  \end{block}

  \begin{block}{Возможность отключения проверки типов}
\begin{verbatim}
unsafeCoerce :: a -> b

cast :: (Typeable a, Typeable b) => a -> Maybe b
cast x = r
  where r = if typeOf x == typeOf (fromJust r)
              then Just $ unsafeCoerce x
              else Nothing
\end{verbatim}
  \end{block}
\end{frame}

\begin{frame}[fragile]
  \frametitle{Системы работы с данными разных типов}
  \framesubtitle{существующие решения}

  \begin{block}{Примеры систем}
    \begin{itemize}
    \item Instance Deriving
    \item Data.Typeable
    \item Data.Dynamic
    \item Generic Haskell
      \begin{itemize}
      \item SYB
      \item RepLib
      \item Uniplate
      \item \dots
      \end{itemize}
    \end{itemize}
  \end{block}
\end{frame}

\begin{frame}[fragile]
  \frametitle{Реализация полиморфных функций}
  \framesubtitle{Data.Dynamic}

  \begin{block}{Пример функции}
\begin{verbatim}
plus :: [Dynamic] -> Maybe Dynamic
plus [a, b]
  | (dynTypeRep a) == (typeOf (undefined :: Integer))
    && (dynTypeRep b) == (typeOf (undefined :: Integer))
  = (fromDynamic a :: Maybe Integer) >>= \a ->
    (fromDynamic b :: Maybe Integer) >>= \b ->
    return $ toDyn (a + b)
  | (dynTypeRep a) == (typeOf (undefined :: Double))
    && (dynTypeRep b) == (typeOf (undefined :: Double))
  = (fromDynamic a :: Maybe Double) >>= \a ->
    (fromDynamic b :: Maybe Double) >>= \b ->
    return $ toDyn (a + b)
plus _ = Nothing
\end{verbatim}
  \end{block}
\end{frame}

\begin{frame}[fragile]
  \frametitle{Генерация функций}
  \framesubtitle{Template Haskell}

  \begin{block}{Пример функции}
\begin{verbatim}
p1 :: Double -> Double -> Double
p1 = (+)

plus :: [Dynamic] -> Maybe Dynamic
plus [a, b] = $(typesAndExecutesN [
  ([('a, [t| Integer |]),
    ('b, [t| Integer |])],
  [| a + b |]),

  ([('a, [t| Double |]),
    ('b, [t| Double |])],
  [| (unsafeCoerce p1 :: (a -> a -> a)) a b |])
])
\end{verbatim}
  \end{block}
\end{frame}

\begin{frame}[fragile]
  \frametitle{Генерация функций}
  \framesubtitle{Метаязык}

  \begin{block}{Пример функции}
\begin{verbatim}
Integer plus (Integer a, Integer b)
Double plus (Double a, Double b)
{
  a + b
}
\end{verbatim}
  \end{block}
\end{frame}

\begin{frame}[fragile]
  \frametitle{Описание системы}
  \framesubtitle{}
  
  \begin{block}{Реализованные модули}
    \begin{itemize}
    \item Haskell библиотека реализующая решение на Template Haskell
    \item Программа генерирующая динамические функции и данные
      необходимые для их совместного использования.
    \item Модуль языка Haskell реализующий примитивы для использования
      динамических функций и простейший цикл ввода-вывода данных
    \item Программа для генерации главного модуля системы.
    \end{itemize}
  \end{block}
\end{frame}
\end{document}
